% =============================================================================
% FILE NAME : thesis.tex
% DEPARTMENT: University of Tuebingen
% AUTOR     : Paul Palomero Bernardo & Konstantin Lübeck
% =============================================================================
% CONTENT   : LaTeX header file for master's thesis
% =============================================================================
% Documentclass and packages
% =============================================================================
\documentclass[oneside,11pt,a4paper,twoside]{scrreprt}
\usepackage[left=3cm, right=3cm,top=3cm,bottom=3.5cm]{geometry}
\usepackage{xspace}[2006/05/08]
\xspaceaddexceptions{])\}}
\usepackage[backend=biber, style=ieee, url=false, doi=false, isbn=false]{biblatex}
\usepackage{scrhack}
\usepackage[utf8]{inputenc}
\usepackage[T1]{fontenc}
\usepackage[autostyle]{csquotes}
\usepackage[ngerman,english]{babel}
\usepackage{mathtools}
\usepackage{amsmath}
\usepackage{amssymb}
\usepackage{amsfonts}
\usepackage{amsthm}
\usepackage{lmodern}
\usepackage[table]{xcolor}
\usepackage{graphicx}
\usepackage{verbatim}
\usepackage[margin=10pt,font=small,labelfont=bf]{caption}
\usepackage{subcaption}
\usepackage{listings}
\usepackage{algorithm}
\usepackage[noend]{algpseudocode}
\usepackage{paralist}
\usepackage{booktabs}
\usepackage{multirow}
\usepackage{siunitx}
\usepackage{pifont}
\usepackage{lscape}
\usepackage{enumitem}
\usepackage{setspace}
\usepackage[toc,acronym,nonumberlist,automake,nopostdot,nogroupskip]{glossaries}
\usepackage{todonotes}
\usepackage{pgfplots}
\usepackage[usenames,dvipsnames]{pstricks}
\usepackage{epsfig}
\usepackage{rotating}
\usepackage{changepage}
\usepackage[noabbrev]{cleveref}
\usepackage{ifthen}
\usepackage{blindtext}
% =============================================================================
% Custom commands
% =============================================================================
% =============================================================================
% Variable declarations
% =============================================================================

% true = english
% false = german
\newboolean{english}
\setboolean{english}{true}

\def\authorsurname{Schammo}
\def\authorfirstname{Tom}
\def\title{Implementation and Analysis of different Input Sources on the Ultratrail Architecture}
\ifthenelse{\boolean{english}}
    {\def\thesiskind{Bachelor's Thesis Computer Science}}
    {\def\thesiskind{Bachelorarbeit Informatik}}
\def\thesisstart{16.11.2022}
\def\thesisend{16.03.2023}
% =============================================================================
% Page margin, header setup, and typography
% =============================================================================

\textwidth 14cm
\textheight 22cm
\topmargin 0.0cm
\evensidemargin 1cm
\oddsidemargin 1cm
\pagestyle{headings}
\clubpenalty10000
\widowpenalty10000
\displaywidowpenalty=10000
\parskip 0.5ex plus 0.1ex minus 0.1ex
\ifthenelse{\boolean{english}}
    {\parindent 12pt}
    {\parindent 0.0cm}

% =============================================================================
% Package setup
% =============================================================================
% === Array ===
\renewcommand{\arraystretch}{1.3}
% === Graphicx ===
\graphicspath{{images/}}
% === Biblatex ===
\addbibresource{includes/references.bib}
\AtEveryBibitem{%
	\clearlist{language}%
}
% === Glossaries ===
\makeglossaries
\renewcommand{\glossarypreamble}{\glsfindwidesttoplevelname[\currentglossary]}
% === Algpseudocode ===
\newcommand{\algorithmautorefname}{Algorithm}
\algrenewcommand{\algorithmiccomment}[1]{\hfill\textit{#1}}
\algdef{S}[FOR]{LFor}[2]{\textit{#1:}\ \algorithmicfor\ #2\ \algorithmicdo}
% === Pgfplots ===
\pgfplotsset{compat=newest}
% === Amsthm ===
\theoremstyle{definition}
\newtheorem{defn}{Definition}[chapter]
\newtheorem{thm}[defn]{Satz}
\newtheorem{analysis}{Analyse}[chapter]
\theoremstyle{remark}
\newtheorem{exmp}[defn]{Beispiel}
\newtheorem*{note}{Bemerkung}
% === xcolor ===
\definecolor{universityred}{RGB}{165,30,55}

% === Listings ===

\ifthenelse{\boolean{english}}
{}
{
    \renewcommand{\lstlistingname}{Quellcodeauszug}
    \renewcommand{\lstlistlistingname}{Quellcodeauszugsverzeichnis}
}

% code colors
\definecolor{darkred}{rgb}{0.6,0.0,0.0}
\definecolor{darkgreen}{rgb}{0,0.50,0}
\definecolor{lightblue}{rgb}{0.0,0.42,0.91}
\definecolor{orange}{rgb}{0.99,0.48,0.13}
\definecolor{grass}{rgb}{0.18,0.80,0.18}
\definecolor{pink}{rgb}{0.97,0.15,0.45}

% General Setting of listings
\lstset{
  aboveskip=1em,
  breaklines=true,
  captionpos=b,
  escapeinside={\%*}{*)},
  frame=single,
  numbers=left,
  numbersep=15pt,
  numberstyle=\tiny,
  literate=%
  {Ö}{{\"O}}1
  {Ä}{{\"A}}1
  {Ü}{{\"U}}1
  {ß}{{\ss}}1
  {ü}{{\"u}}1
  {ä}{{\"a}}1
  {ö}{{\"o}}1
  {~}{{\textasciitilde}}1
}
% 0. Basic Color Theme
\lstdefinestyle{colored}{ %
  basicstyle=\ttfamily,
  backgroundcolor=\color{white},
  commentstyle=\color{green}\itshape,
  keywordstyle=\color{blue}\bfseries\itshape,
  stringstyle=\color{red},
}
% 1. General Python Keywords List
\lstdefinelanguage{PythonPlus}[]{Python}{
  morekeywords=[1]{,as,assert,nonlocal,with,yield,self,True,False,None,} % Python builtin
  morekeywords=[2]{,__init__,__add__,__mul__,__div__,__sub__,__call__,__getitem__,__setitem__,__eq__,__ne__,__nonzero__,__rmul__,__radd__,__repr__,__str__,__get__,__truediv__,__pow__,__name__,__future__,__all__,}, % magic methods
  morekeywords=[3]{,object,type,isinstance,copy,deepcopy,zip,enumerate,reversed,list,set,len,dict,tuple,range,xrange,append,execfile,real,imag,reduce,str,repr,}, % common functions
  morekeywords=[4]{,Exception,NameError,IndexError,SyntaxError,TypeError,ValueError,OverflowError,ZeroDivisionError,}, % errors
  morekeywords=[5]{,ode,fsolve,sqrt,exp,sin,cos,arctan,arctan2,arccos,pi, array,norm,solve,dot,arange,isscalar,max,sum,flatten,shape,reshape,find,any,all,abs,plot,linspace,legend,quad,polyval,polyfit,hstack,concatenate,vstack,column_stack,empty,zeros,ones,rand,vander,grid,pcolor,eig,eigs,eigvals,svd,qr,tan,det,logspace,roll,min,mean,cumsum,cumprod,diff,vectorize,lstsq,cla,eye,xlabel,ylabel,squeeze,}, % numpy / math
}
% 2. New Language based on Python
\lstdefinelanguage{PyBrIM}[]{PythonPlus}{
  emph={d,E,a,Fc28,Fy,Fu,D,des,supplier,Material,Rectangle,PyElmt},
}
% 3. Extended theme
\lstdefinestyle{colorEX}{
  basicstyle=\small\ttfamily,
  backgroundcolor=\color{white},
  commentstyle=\color{darkgreen}\slshape,
  keywordstyle=\color{blue}\bfseries\itshape,
  keywordstyle=[2]\color{blue}\bfseries,
  keywordstyle=[3]\color{grass},
  keywordstyle=[4]\color{red},
  keywordstyle=[5]\color{orange},
  stringstyle=\color{darkred},
  emphstyle=\color{pink}\underbar,
}

% =============================================================================
% siunitx
% =============================================================================
\sisetup{locale = UK}
\sisetup{detect-all = true}
\sisetup{per-mode = symbol}
\DeclareSIUnit\op{OP}
\DeclareSIUnit\ops{OPS}
\DeclareSIUnit\mac{MAC}
\DeclareSIUnit\macs{MACS}
\DeclareSIUnit\cycle{cycle}
% =============================================================================
% symbols
% =============================================================================
\newcommand{\cmark}{\ding{51}}%
\newcommand{\xmark}{\ding{55}}%
% =============================================================================
% tabular
% =============================================================================
\newcommand{\rowgray}{\rowcolor[gray]{.9}}%
% =============================================================================
% mathtools: equation numbering style
% =============================================================================
\renewcommand{\theequation}{\arabic{chapter}.\arabic{equation}}
\newtagform{bold}[\textbf]{}{}
\usetagform{bold}
% =============================================================================
% enumitem
% =============================================================================
\setitemize{itemsep=0.15em, topsep=0.5em}
\setenumerate{itemsep=0.15em, topsep=0.5em}

% =============================================================================
% FILE NAME : abbreviations.tex
% DEPARTMENT: University of Tuebingen
% AUTOR     : Paul Palomero Bernardo
% =============================================================================
% CONTENT   : Include for abbreviations
% =============================================================================
% =============================================================================
% DOCUMENT START
% =============================================================================
\begin{document}
% =============================================================================
% Titlepage
% =============================================================================
\ifthenelse{\boolean{english}}{
	\newgeometry{top=2.7cm,left=2.5cm,right=2.5cm}
	\begin{titlepage}
		\begin{adjustwidth}{0cm}{-1cm}
		\begin{minipage}{0.5\textwidth}
			\includegraphics[width=\linewidth]{figures/logo.pdf}
		\end{minipage}
		\hfill
		\begin{minipage}{0.3\textwidth}
			{\bfseries\textsf{\textcolor{universityred}{Faculty of\\Science}}\\[0.3cm]}
			{\bfseries\textsf{\textcolor{universityred}{Embedded Systems}}}
		\end{minipage}%
		\end{adjustwidth}
		\vspace{3cm}
		\begin{center}
			{\huge \thesiskind\\[2cm]}
			{\Large\bfseries \title\\[1.5cm]}
			{\large Submitted by: \authorfirstname\ \authorsurname}\\[0.5cm]
			\thesisend
		\end{center}
		\vfill
		\begin{minipage}{0.48\textwidth}
			\centering
			{\small\bfseries First Supervisor}\\[0.3cm]
			{\large Prof. Dr. Oliver Bringmann}\\[2mm]
			{\footnotesize Faculty of Science\\
			Department of Computer Science\\[1.5mm]
			Embedded Systems\\
			University of Tübingen}
		\end{minipage}
		\hfill
		\begin{minipage}{0.48\textwidth}
			\centering
			{\small\bfseries Graduate Advisor}\\[0.3cm]
			{\large Christoph Gross}\\[2mm]
			{\footnotesize Faculty of Science\\
			Department of Computer Science\\[1.5mm]
			Embedded Systems\\
			University of Tübingen}
		\end{minipage}%
		\vspace{1.5cm}
		\begin{minipage}{0.48\textwidth}
			\centering
			{\small\bfseries Graduate Advisor}\\[0.3cm]
			{\large Adrian Frischknecht}\\[2mm]
			{\footnotesize Faculty of Science\\
				Department of Computer Science\\[1.5mm]
				Embedded Systems\\
				University of Tübingen}
		\end{minipage}
		\hfill
		\begin{minipage}{0.48\textwidth}
			\centering
			{\small\bfseries Graduate Advisor}\\[0.3cm]
			{\large Paul Palomero Bernardo}\\[2mm]
			{\footnotesize Faculty of Science\\
				Department of Computer Science\\[1.5mm]
				Embedded Systems\\
				University of Tübingen}
		\end{minipage}
	\end{titlepage}
}{
	\newgeometry{top=2.7cm,left=2.5cm,right=2.5cm}
	\begin{titlepage}
		\begin{adjustwidth}{0cm}{-1cm}
		\begin{minipage}{0.5\textwidth}
			\includegraphics[width=\linewidth]{figures/logo.pdf}
		\end{minipage}
		\hfill
		\begin{minipage}{0.4\textwidth}
			{\bfseries\textsf{\textcolor{universityred}{Mathematisch-\\Naturwissenschaftliche Fakultät}}\\[0.3cm]}
			{\bfseries\textsf{\textcolor{universityred}{Eingebettete Systeme}}}
		\end{minipage}%
		\end{adjustwidth}
		\vspace{3cm}
		\begin{center}
			{\huge \thesiskind\\[2cm]}
			{\Large\bfseries \title\\[1.5cm]}
			{\large \authorfirstname\ \authorsurname}\\[0.5cm]
			\thesisend
		\end{center}
		\vfill
		\begin{minipage}{0.48\textwidth}
			\centering
			{\small\bfseries Erster Gutachter}\\[0.3cm]
			{\large Prof. Dr. Oliver Bringmann}\\[2mm]
			{\footnotesize Mathematisch-Naturwissenschaftliche\\Fakultät\\
			Fachbereich Informatik\\[1.5mm]
			Lehrstuhl Eingebettete Systeme\\
			Universität Tübingen}
		\end{minipage}
		\hfill
		\begin{minipage}{0.48\textwidth}
			\centering
			{\small\bfseries Zweiter Gutachter}\\[0.3cm]
			{\large TODO: NAME}\\[2mm]
			{\footnotesize Mathematisch-Naturwissenschaftliche\\Fakultät\\
			Fachbereich Informatik\\[1.5mm]
			TODO: LEHRSTUHL\\
			Universität Tübingen}
		\end{minipage}%
		\vspace{1.5cm}
		\begin{minipage}{0.48\textwidth}
			\centering
			{\small\bfseries Betreuer}\\[0.3cm]
			{\large TODO: NAME}\\[2mm]
			{\footnotesize Mathematisch-Naturwissenschaftliche\\Fakultät\\
			Fachbereich Informatik\\[1.5mm]
			Lehrstuhl Eingebettete Systeme\\
			Universität Tübingen}
		\end{minipage}
		\hfill
		\begin{minipage}{0.48\textwidth}
			\centering
			{\small\bfseries Betreuer}\\[0.3cm]
			{\large TODO: NAME}\\[2mm]
			{\footnotesize Mathematisch-Naturwissenschaftliche\\Fakultät\\
			Fachbereich Informatik\\[1.5mm]
			Lehrstuhl Eingebettete Systeme\\
			Universität Tübingen}
		\end{minipage}
	\end{titlepage}
}
\restoregeometry

% =============================================================================
% 2nd page: Titlepage
% =============================================================================
\thispagestyle{empty}
\vspace*{\fill}
\noindent
\begin{minipage}{.8\textwidth}
	\textbf{\authorsurname, \authorfirstname:}\\
	\emph{\title}\\
	\thesiskind\\
	\ifthenelse{\boolean{english}}{
		University of Tübingen\\
		Processing period: \thesisstart\ - \thesisend
	}{
		Eberhard Karls Universität Tübingen\\
		Bearbeitungszeitraum: \thesisstart\ - \thesisend
	}
\end{minipage}
\cleardoublepage

% =============================================================================
% Abstract
% =============================================================================
\begin{onehalfspace}
\ifthenelse{\boolean{english}}{
	\chapter*{Kurzfassung}
	\thispagestyle{empty}
	\begin{otherlanguage}{ngerman}
		% =============================================================================
% FILE NAME : abstractger.tex
% DEPARTMENT: University of Tuebingen
% AUTOR     : Tom Schammo
% =============================================================================
% CONTENT   : Include for abstract (german)
% =============================================================================

Ziel dieser Arbeit ist die Implementierung von Keyword-Spotting auf der UltraTrail-Architektur.
Zu dem Zweck werden verschiedene Eingabequellen implementiert und analysiert.
Ein weiteres Ziel ist es, das Rust-Ökosystem zu erweitern, indem die Unterstützung für UltraTrail zu Rust hinzugefügt wird.
Um dies zu erreichen, wird UltraTrail zur PAC hinzugefügt und die Treiber werden in der HAL implementiert.
Die $I^2S$- und PDM-Treiberimplementierungen werden dann getestet.
Die Tests zeigen, dass die Mikrofontreiber nicht wie erwartet funktionieren und trotz der Behebung einiger Probleme war es nicht möglich, die Mikrofone wie gewünscht zum Laufen zu bringen, daher konnte Keyword-Spotting nicht implementiert werden.
Der UltraTrail-Treiber wurde jedoch erfolgreich implementiert und getestet, so dass
die Wirksamkeit des Keyword-Spotting auf dem UltraTrail nun weiter erforscht werden kann.

	\end{otherlanguage}
	\cleardoublepage
	\chapter*{Abstract}
	\thispagestyle{empty}
	% =============================================================================
% FILE NAME : abstract.tex
% DEPARTMENT: University of Tuebingen
% AUTOR     : Tom Schammo
% =============================================================================
% CONTENT   : Include for abstract (english)
% =============================================================================
This thesis sets out to implement keyword-spotting on the UltraTrail architecture.
One additional goal is extending the Rust ecosystem by adding support for UltraTrail to Rust.
To achive this, UltraTrail is added to the PAC and its drivers are implmented in the HAL.
Then the implementation of the $I^2S$ and PDM drivers are tested.
Testing reveals that the microphone drivers do not work as expected.
While fixing a few bugs, it was not possible to get the microphones working as
desired, so keyword-spotting could not be implemented.
However the UltraTrail driver has successfully been implemented and tested.

	\cleardoublepage
}{
	\chapter*{Kurzfassung}
	\thispagestyle{empty}
	% =============================================================================
% FILE NAME : abstract.tex
% DEPARTMENT: University of Tuebingen
% AUTOR     : Tom Schammo
% =============================================================================
% CONTENT   : Include for abstract (english)
% =============================================================================
This thesis sets out to implement keyword-spotting on the UltraTrail architecture.
One additional goal is extending the Rust ecosystem by adding support for UltraTrail to Rust.
To achive this, UltraTrail is added to the PAC and its drivers are implmented in the HAL.
Then the implementation of the $I^2S$ and PDM drivers are tested.
Testing reveals that the microphone drivers do not work as expected.
While fixing a few bugs, it was not possible to get the microphones working as
desired, so keyword-spotting could not be implemented.
However the UltraTrail driver has successfully been implemented and tested.

	\cleardoublepage
}
\end{onehalfspace}

% =============================================================================
% Acknowledgements
% =============================================================================
\begin{onehalfspace}
\ifthenelse{\boolean{english}}
	{\chapter*{Acknowledgements}}
	{\chapter*{Danksagung}}
\thispagestyle{empty}
Danksagung (falls gewünscht)
\cleardoublepage
\end{onehalfspace}

% =============================================================================
% Table of contents
% =============================================================================
% page numbering setup
\pagenumbering{Roman}
\setcounter{page}{1}
\pagestyle{headings}
\renewcommand{\baselinestretch}{1.3}
\small\normalsize
\tableofcontents
\renewcommand{\baselinestretch}{1}
\small\normalsize
\cleardoublepage
% =============================================================================
% Page numbering setup
% =============================================================================
\pagenumbering{arabic}
\setcounter{page}{1}

% =============================================================================
% Chapters
% =============================================================================
\begin{onehalfspace}
\ifthenelse{\boolean{english}}
	{\chapter{Introduction}}
	{\chapter{Einleitung}}
\label{cha:introduction}
% =============================================================================
% FILE NAME : 00_introduction.tex
% DEPARTMENT: University of Tuebingen
% AUTOR     : Tom Schammo
% =============================================================================
% CONTENT   : Include for chapter "Introduction"
% =============================================================================


Rust \cite{rustlang} is a somewhat new language,
with their very first release in early 2012 \cite{rust_releases}.
But the 1.0 alpha release was only in early 2015 \cite{rust_releases}
with the full 1.0 release following a few months later in Mai of the same year \cite{rust_releases}.
C, as a comparison has been used as early as the 1970s.\\
But Rust has been gaining in popularity \cite{rust_popularity} over the last few years, however
due to its relatively 'young' age there still are huge gaps when it comes to software support and available libraries.
Embedded systems is an area where Rust has the potential to be very useful due to it being very performant as well as memory safe,
thereby being a potential alternative to C or C++, especially in high stakes, high performance, real time applications.
However, that gap is very noticeable when it comes to embedded devices.\\
Additionally, with the improvement of machine learning technologies and their incorporation into IoT- and smart devices,
the use cases for hardware accelerated devices are ever-growing.
As voice controlled devices, specifically voice assistants (VAs) have become almost omnipresent,
with Siri being shipped with every Apple smartphone, tablet and laptop, Google Assistant being installed
on every Android phone (except for a few custom ROMs that remove Google software) and Cortana being integrated
into every new Windows operating system, users become increasingly used to these technologies.
The adoptions of voice controlled devices can also be observed in private homes with standalone
devices like the Amazon Echo to control appliances and a variety of IoT gadgets.
However, in the IoT space, security is more often than not, not even an afterthought.
This is mostly a result of software bugs that could be avoided by enforcing memory safety.
So in my thesis I'll improve upon the Rust ecosystem by expanding upon the thesis of Raphael Vogelgsang \cite{rust_pulp}
and implementing support for the UltraTrail \cite{ultratrail} AI accelerator.
\newpage
This thesis is structured as follows:
The chapter \ref{cha:fundamentals} will provide a fundamental overview of RISC-V, 'Embedded AI', (embedded) Rust and microphone technology.\\
Chapter \ref{cha:related_work} covers previous work on the PULPissimo, and goes a bit into the UltraTrail architecture.
Finally, it covers the state of keyword spotting in the industry.
It briefly goes into its use cases and then covers one of them a little more in depth.
The concept of the thesis is covered in \ref{cha:concept}.
That chapter briefly introduces what I'm trying to do in my thesis.
After that it covers the microphones used and finally goes into the tests that I have used
to assess the functionality of the UltraTrail and microphone drivers.
\todo{rest der kapitel}
It is followed by chapter \ref{cha:results_and_discussion}, which first discuses the implementation
of the UltraTrail driver as well as the PDM and $I^2S$ microphone.
After that, it will introduce the tests that have been performed to ensure the functionality of the drivers,
followed by a brief discussion about the results.
Finally chapter \ref{cha:conclusion_and_future_work} concludes this thesis and goes a bit into \todo{goes into what?}


% \section{Umfang der Arbeit}
% ,,Dies ist eine der meistgestellten Fragen. Natürlich verbirgt sich dahinter die Vermutung, die erzielbare Note sei – gutachterabhängig – mit der Seitenzahl korreliert (vgl. Abb.~\ref{fig:graph}). Nur
% wie? Linear, normalverteilt, nach dem Gesetz vom abnehmenden Grenznutzen?
% Tatsächlich kommt es auf die Qualität Ihrer Resultate an. Wenn Sie mit Ihrer Arbeit das Collatz-Problem, auch bekannt als Ulams Vermutung, widerlegen können, genügt eine Seite Inhalt mit dem Hinweis, die Zahl, welche die Vermutung widerlegt, befinde sich auf der beigefügten CD.
%
% \begin{figure}[htb]
%   \centering
%   \includegraphics[width=0.9\textwidth]{figures/note_page.png}
%   \caption[Verteilung: Seitenanzahl-Note]{Welcher Verteilung folgt die Note als Funktion der Seitenzahl?}
%   \label{fig:graph}
% \end{figure}
%
% Für alle, die nicht so viel Glück haben, soll die folgende Tabelle \ref{tab:tabelle-2} auf Seite \pageref{tab:tabelle-2} als Richtschnur
% dienen. Dabei wurden Anhänge, Inhalts- und Abbildungsverzeichnisse sowie Stichwortverzeichnisse (sofern überhaupt vorhanden, da nicht üblich) nicht gerechnet.Bedenken Sie, dass Ihr Gutachter das alles gründlich lesen soll, der Zweitgutachter es vielleicht
% auszugsweise lesen muss. Formulieren Sie deshalb knapp und auf den Punkt, vermeiden Sie Wiederholungen („Wir kommen nochmal auf das schwierige Problem der Softwareauswahl aus
% Kapitel zwei zu sprechen, wo wir feststellten, dass ...“).
% Längliche Passagen, etwa Programmstücke, Teile der Dokumentation, sehr lange Zitate (etwa ein Beweis, ein Gerichtsurteil, ein Zeitschriftenartikel im Wortlaut), Messreihen usw.
% verbannen Sie in den Anhang (mit der Gewissheit, dass das kaum jemand gründlich lesen wird).
% Aber auch bei den Anhängen ist weniger oft mehr. Noch umfangreichere Teile lassen sich auf
% eine CD brennen, die der Arbeit beigefügt wird; allerdings ist umstritten, ob ein Gutachter sich
% diese anschauen muss.
%
% \begin{table}[tb]
%   \centering
%   \begin{tabular}{cccc}
%     \toprule
%     \textbf{Art der Arbeit} & \textbf{Untergrenze} & \textbf{Obergrenze} & \textbf{Anmerkung} \\
% 		\midrule
%     Bachelor & 35 & 65 & ideal $\leq$ 50 \\
%     Master & 50 & 85 & ideal $\leq$ 70 \\
%     \bottomrule
%   \end{tabular}
%   \caption{Empfehlung zur Seitenanzahl der Arbeit}
%   \label{tab:tabelle-2}
% \end{table}
%
%
% Weil das Vorwort, der erste Abschnitt der Einleitung und die abschließende Zusammenfassung mit Ausblick immer gründlich gelesen werden, sollten Sie darauf besonderes Augenmerk
% legen. In der Regel schreibt man die Einleitung und das Vorwort auch erst, wenn der restliche
% Teil einschließlich Zusammenfassung (Fazit) steht, Spötter nennen das die Anpassung des
% Anforderungsprofils an das tatsächlich erzielte Resultat.
% Zuletzt ein Rat, wenn der Umfang der Arbeit erkennbar zu groß wird. So wie bei Seminarvorträgen Schnellersprechen das Problem eines zu umfangreichen Folienprogramms nicht
% lösen kann, so wenig lässt sich mit typografischen Mitteln (kleinerem Font, engeren Zeilenabständen, breiteren Spalten) wesentlich Platz ohne Verlust an Lesbarkeit gewinnen. Sie kommen
% nicht umhin, größere Teile der Arbeit zu streichen oder wesentlich zu straffen.
% Dafür bieten sich oft die Kapitel an, in denen Sie den mühsamen Prozess der Lösungsfindung
% einschließlich aller notwendigen Vorarbeiten und Diskussionen mit dem Anwender dokumentiert haben. Hinter solchen längeren Beschreibungen steckt der verständliche Wunsch, der Gutachter möge honorieren, dass Sie unglaublich mit dem Auftraggeber, der undurchsichtigen
% Software, dem abstürzenden Computer u.a.m. kämpfen mussten und vieles zunächst nicht so
% funktionierte, wie gedacht.
% Leser sind aber wie Restaurantgäste, Gutachter ähneln Gourmetkritikern. Sie sind mitleidslos und schauen nur auf den Teller vor sich. Sie wollen nichts davon wissen, dass frische Seezunge heute enorm schwierig zu beschaffen war und der Jungkoch sich am Gratin die Finger
% verbrannt hat. Halten Sie Ihre Schwierigkeiten in einem ehrlich geschriebenen 10-Zeilen-Abschnitt der Zusammenfassung fest, als Teil der Selbstreflektion, die immer zu einer
% Abschlussarbeit gehört, und streichen Sie schweren Herzens Teile der Entwicklungssaga.''

\cleardoublepage

\ifthenelse{\boolean{english}}
	{\chapter{Fundamentals}}
	{\chapter{Grundlagen}}
\label{cha:fundamentals}
% =============================================================================
% FILE NAME : 01_fundamentals.tex
% DEPARTMENT: University of Tuebingen
% AUTOR     : Tom Schammo
% =============================================================================
% CONTENT   : Include for chapter "Fundamentals"
% =============================================================================


\section{RISC-V}

RISC-V \cite{riscv} is an open standard Instruction Set Architecture (ISA), which
was originally developed to support computer architecture research and education.
However, the authors now aim for RISC-V to also be used in industry implementations \cite{riscv_spec}.

\subsection{ISA}

A computer program consists of several instructions.
These instructions are 'commands' that a computer understands and can react to by performing certain tasks in response.
A compiler, in case of compiled languages like C, C++, Rust or Go, transforms code written by a human into instructions.
In case of interpreted languages like Python, Java or JavaScript, the source code is converted into these instructions
(also sometimes referred to as 'machine code') by an interpreter or Just-In-Time (JIT) compiler during the programs' execution.
Sometimes that code is first transformed into bytecode which serves as an intermediate representation of the source code
that can be optimized.
Java, for example is first compiled into bytecode that can be distributed and executed by the Java Virtual Machine (JVM),
whereas the python source code is often interpreted immediately line by line.
\\
An Instruction Set is, as the name suggests, the set of instructions that a specific computer, or to be more precise,
its Central Processing Unit (CPU) can understand.
Which instructions are available depends on the hardware of a particular system.
Instruction Set Architectures abstractly describe the architecture of a computer,
like supported data types, available registers, how main memory is managed by the hardware and
which instructions a microprocessor can execute \cite{isa}, which can then be implemented by a CPU.
ISAs are often classified by their complexity, belonging either to the set of 'complex instruction set computers' (CISC), like x86,
or 'reduced instruction set computers' (RISC) like ARM \cite{arm_architecture} or RISC-V \cite{riscv_spec}.

\subsection{RISC-V vs. ARM}

While RISC-V \cite{riscv} is an open standard ISA provided, royalty-free, by 'RISC-V International', a non-profit,
ARM \cite{arm} is developed by a company that sells licenses to other businesses that develop CPUs based on the ARM architecture.
ARM generally has a much higher market share than RISC-V due to it being used in pretty much every mobile device (phones, tablets, smartwatches)
these days, as well as many IoT devices and even laptops like the new Apple MacBook containing the M1 system-on-a-chip (SOC).
However, NVIDIA announced its acquisition of ARM in 2020 \cite{arm_sale}, which was met with disapproval by some people.
This caused speculations, that some companies might turn to RISC-V in the future as an alternative \cite{arm_sale_speculation}.
However, the acquisition was called off in 2022 \cite{arm_sale_called_off}.\\\\
The advantage of RISC-V lies in the open nature of the standard that allows anyone to use it without paying royalties, while still
having the option to build upon the open standard.
So companies can still have their own (open or proprietary) solutions and implementations \cite{riscv_about} without having to pay for
the usage of RISC-V.

\section{Embedded AI}

With artificial intelligence (AI) becoming more and more contemporary and wide-ranging it is only consequential that is has found its way into embedded devices.
Especially with the rising trend of IoT devices and smart homes, embedded AI has found its way into the hand of consumer homes.
From voice assistants like Amazon's Alexa \cite{alexa} over autonomous robots like the Roomba \cite{roomba} from iRobot to self-driving cars like Tesla's autopilot \cite{autopilot}.
However size and power requirements of those devices severely limit their capabilities, which is why the development of specialized hardware to improve performance
and power usage becomes a necessity.\\\\
In my thesis I'm working with the UltraTrail TC-ResNet AI Accelerator \cite{ultratrail}.
An AI Accelerator is a type of hardware accelerator specialized for artificial intelligence (AI) and machine learning (ML) applications, such as neural networks (NN).
A hardware accelerator is a set of hardware that specializes in carrying out a specific set of tasks really well while sacrificing generality.
It is created to perform solve one type of problem faster and/or more efficiently than a generic CPU could with the trade-off that it may not be able to do other things at all.
Examples include, but are not limited to, GPUs, sound cards, cryptographic accelerators, or, in this case, AI accelerators.\\\\


\section{Rust}

The software written for this thesis, that is running on the PULPissimo \cite{pulpissimo} has been implemented using the Rust programming language \cite{rustlang}.

\subsection{Why Rust?}

While C is the industry standard for writing software for embedded systems, along with C++ being fairly popular as well, Rust has a couple of
advantages over those languages.\\\\
The biggest downside of C is likely the lack of memory safety, which can not only cause software to be less reliable, but
can also make a system vulnerable to different attacks, such as buffer overflows, reading uninitialized variables or use-after-free, which can compromise cybersecurity \cite{memory_safety}.\\
C++ fixes some of those issues due to the availability of smart pointers, but those are fairly 'modern' C++ features, thus being only available if compilers support more recent
versions of C++ (with C++ 11 starting to introduce many new important concepts \cite{cpp11}).
There is also no guarantee that those features have been used in any given program.
The opposite is actually more likely due to the incorporation of older software (library code written using older C++ versions),
compilers not supporting new versions (especially likely with small vendors that don't have the resources to continuously update compilers for their platform),
or developers not keeping up to date with new language features and therefore not being aware of or able to use them.\\\\
Rust on the other hand can (mostly) give compile time guarantees on the software's safety and reliability.

% NOTE: fixes weird spacing on the next page that is introduced by the '\newpage'
\raggedbottom
\newpage

\subsection{Memory safety and performance}

\subsubsection{Garbage collection}

Most programming languages that try to achieve memory safety use a garbage collector for that purpose \cite{java_garbage_collector}.
A garbage collector first identifies pieces of memory that are no longer used, illustrated in figure \ref{fig:gc_mark}.
Then the marked pieces of memory are periodically deleted. Figure \ref{fig:gc_delete} describes how that same chunk of memory would look like after marked sections are deleted.
Finally, after the deletion, the garbage collector will 'compact' the memory to avoid memory fragmentation and improve the speed of future allocations as shown in figure \ref{fig:gc_compact}.

\begin{figure}[htb]
    \centering
    \includegraphics[width=0.9\textwidth]{figures/fundamentals_garbage_collector_marking.PNG}
    \caption[Illustration: Garbage Collector marking memory for deletion \cite{java_garbage_collector}]{Garbage Collector marking memory for deletion}
    \label{fig:gc_mark}
\end{figure}

\begin{figure}[htb]
    \centering
    \includegraphics[width=0.9\textwidth]{figures/fundamentals_garbage_collector_deletion.PNG}
    \caption[Illustration: Garbage Collector deleting marked memory \cite{java_garbage_collector}]{Memory layout after deletion of unused pieces of memory}
    \label{fig:gc_delete}
\end{figure}

\begin{figure}[H]
    \centering
    \includegraphics[width=0.9\textwidth]{figures/fundamentals_garbage_collector_compacting.PNG}
    \caption[Illustration: Garbage Collector compacting memory \cite{java_garbage_collector}]{Memory layout after compacting}
    \label{fig:gc_compact}
\end{figure}

Garbage collectors however have a few drawbacks, like unpredictable latency (the running program has to be stopped while the garbage collector is running) and a hit in performance.
The runtime of the programming language, that contains the garbage collector, also needs to be installed on the target system and run alongside the program, using additional
resources which can be problematic in a resource scarce environment like small embedded devices.
Therefore, garbage collected languages are rarely used in embedded systems programming.
Rust uses a different, and unique approach to guarantee memory safety without suffering from any of the aforementioned drawbacks.

\subsubsection{Ownership in Rust}

Rust uses an 'ownership' model \cite{rust_ownership} to make memory safety guarantees at compile time and without the need of a garbage collector,
which is good for safety, and does not negatively impact performance or latency.
This model consists of 3 simple rules:
\begin{enumerate}
    \item Each value has an owner
    \item Each value can only have one owner at a time
    \item When the owner goes out of scope, the value will be dropped
\end{enumerate}

These 'owners' are generally variables. \ref{code:owner} presents a simple example of a 'value' ('\lstinline{5}') and its
'owner' ('\lstinline{owner}').\\

\begin{lstlisting}[style=colorEX,language=Rust,caption={Simple example of a value and it's owner},label={code:owner}]
{
    let owner = 5;
}
\end{lstlisting}

It doesn't matter that another 'owner' might have the same value. In the code depicted in \ref{code:two_owners}
'\lstinline{owner}' and '\lstinline{owner2}' have the same value, but they are different variables, just like in any
other programming language and have nothing to do with each other. They are different 'owners'.\\

\begin{lstlisting}[style=colorEX,language=Rust,caption={Simple example of two owners},label={code:two_owners}]
{
    let owner = 5;
    let owner2 = 5;
}
\end{lstlisting}


If one variable is assigned to another, one of two things happens; if the value is saved on the stack like in
example \ref{code:copy} the value is simply copied to the second variable, and now '\lstinline{owner}' and '\lstinline{owner2}'
hold the same value, but other than that have nothing to do with each other. They are different 'owners'.\\

\begin{lstlisting}[style=colorEX,language=Rust,caption={Simple example of a copy},label={code:copy}]
{
    let owner = 5;
    let owner2 = owner;
}
\end{lstlisting}

\begin{lstlisting}[style=colorEX,language=Rust,caption={Simple example of a move},label={code:move}]
{
    let owner = String::from("value");
    let owner2 = owner;
}
\end{lstlisting}

If the value is stored on the heap like in example \ref{code:move}, the value is not copied, because that could
be potentially very expensive.
So only the pointer (memory address), size of the data, and the capacity of the memory block is copied.
But now '\lstinline{value}' has two owners, which violates rule 2.
This rule is important because due to rule 3:
\begin{enumerate}
    \item If both variables go out of scope at the same time, the memory would be freed twice.
    \item If only one goes out of scope, the other holds the address of invalid (freed) memory.
\end{enumerate}
To avoid those things '\lstinline{value}' is considered to be 'moved' to its new owner, the variable '\lstinline{owner2}'.
The original owner ('\lstinline{owner}') is no longer valid.\\
This works similarly for functions, and these 3 rules allow the compiler to check for any mistakes that would lead to
a double free, use-after-free or memory that has forgotten to be freed.
However, there is one more important aspect of memory management in Rust; the borrow operator.
The borrow operator ('\lstinline{&}') \cite{rust_borrow} allows other variables access to a value without changing the owner.
There are two more rules for references and borrowing:
\begin{enumerate}
    \item There can be an any number of immutable references to a value.
    \item There can never be more than one mutable reference to a value.
\end{enumerate}
The code snippet \ref{code:borrow} displays a valid use of rule 1 and code snippet \ref{code:mut_borrow} gives both examples
of what is valid and invalid under rule 2.

\begin{lstlisting}[style=colorEX,language=Rust,caption={Simple example of an immutable borrow},label={code:borrow}]
{
    let owner = 5;
    let borrower1 = &owner;
    let borrower2 = &owner;
}
\end{lstlisting}

\begin{lstlisting}[style=colorEX,language=Rust,caption={Simple example of a mutable borrow},label={code:mut_borrow}]
{
    let owner = 5;
    let borrower = &mut owner;

    // These 2 statements are not allowed, as the value as already borrowed as mutable.
    // let borrower1 = &owner;
    // let borrower2 = &mut owner;
}
\end{lstlisting}

\subsection{Embedded Rust}

Developing software that runs directly on hardware differs slightly from 'normal' software due to the lack of an operating system (OS).
When developing for a microcontroller (MCU) there is no OS that supervises and manages programs or facilitates communication with peripherals.
Therefore, embedded Rust programs look a little different from those that would run on a Windows, Linux or Mac computer.
\\
First, since there is no operating system to allocate memory, put the code into the right memory space and start the program.
So the developer has to take care of that when creating the executable, which is why 'special' instructions for the linker are necessary.
For Rust programs these instructions are written into a \emph{memory.x} file.
This file contains information about the memory layout of the hardware that the program will be running on.
An example of such a \emph{memory.x} file is displayed in \ref{code:memory_x}.

\newpage
\begin{lstlisting}[style=colorEX,caption={Example memory.x file},label={code:memory_x}]
MEMORY
{
  /* NOTE K = KiBi = 1024 bytes */
  RAM : ORIGIN = 0x1C000000, LENGTH = 0x00010000
  LTWO : ORIGIN = 0x1C010000, LENGTH = 0x00072000
}

SECTIONS
{
    .l2_data ORIGIN(LTWO) :
    {
        LONG(0x00072000);
        *(.l2_data);
    } > LTWO
}

REGION_ALIAS("FLASH", RAM);

REGION_ALIAS("REGION_TEXT", FLASH);
REGION_ALIAS("REGION_RODATA", FLASH);
REGION_ALIAS("REGION_DATA", RAM);
REGION_ALIAS("REGION_BSS", RAM);
REGION_ALIAS("REGION_HEAP", RAM);
REGION_ALIAS("REGION_STACK", RAM);

\end{lstlisting}


However, there are also quite a few differences in the actual program code.
Usually Rust programs contain a main function that looks like the example in \ref{code:os_main}.
\begin{lstlisting}[style=colorEX,language=Rust,caption={Standard main function in Rust},label={code:os_main}]
fn main() {
    // contents of main function
}
\end{lstlisting}
This works similar to normal C code where the main function either serves as entry point for the program or will be called by the '\lstinline{_start}' function provided by \emph{glibc} \cite{before_main}.
This '\lstinline{_start}' function initializes the program runtime by setting up things like the stack and writing arguments into memory to name a few.
After initialization, the main function is then called.
This is equal for programs that run with or without an OS.\\
The difference in this startup process occurs before the '\lstinline{_start}' function.
If the program runs on a device with an OS, the OS first does some preparation and initialization for the program to run.
The program environment is configured by, among other things, checking permissions (is the user allowed to run this program),
allocating space on the stack and heap, initializing the stack pointer, linking dynamically linked libraries and calling pre-initialization functions \cite{before_main}.
Once the program environment is configured, the programs '\lstinline{_start}' is called by the OS.


When a program is running on 'bare-metal' (meaning without an OS or even a bootloader that launches the  application), this process is more crude.
The '\lstinline{_start}' function still serves as the entry point for the program, but there are fewer steps involved in getting there.
After the application is compiled by the programmer, it is 'flashed' onto the device, meaning written to flash memory.
When the processor is powered on, it will copy the program into Random Access Memory (RAM) and jump to the reset interrupt vector address.
The programs reset interrupt handler will then initialize the system and configure any necessary hardware components
like registers, external RAM or the Memory Management Unit (MMU).
After that the handler jumps to '\lstinline{_start}' (if present) \cite{before_main}.\\\\
Due to the lack of an OS, there is no cleanup after the program ended either.
As a matter of fact, since the program will be the only thing running on the hardware,
the main function will never return, for our intents and purposes.
The program also only really stops when the device is turned off or reset.

\begin{lstlisting}[style=colorEX,language=Rust,caption={Example main function for the pulp-platform},label={code:embedded_main}]
#![no_main]

use riscv_rt::entry;

#[entry]
fn main() -> ! {
    // code
    loop {}
}
\end{lstlisting}

\ref{code:embedded_main} displays a main function alike to the ones I've used in my programs.
The '\lstinline{#![no_main]}' directive tells the compiler that I will not use the standard main function.
Since this function will never return, the return type needs to specify exactly that.
In Rust, return types are specified using the '\lstinline{->}', followed by the type \cite{rust_return}.
Replacing the type with a '\lstinline{!}' indicates that the function doesn't return \cite{rust_never_type}.
\\
The '\lstinline{#[entry]}' attribute, imported by the '\lstinline{use riscv_rt::entry}' statement, is used to declare the entry point of the program \cite{riscv_rt_entry}
The requirement for this attribute is that the function never returns, which also why the loop at the end is necessary.
\\\\
This is close to a working program, but not exactly.
Without the OS, the device is also not able to use the standard library.
Which is why the '\lstinline{#![no_std]}' directive is necessary, as it prevents Rust from loading the standard library \cite{rust_no_std}.
However, the standard library provides a lot of useful features, like a dynamic memory allocator or a panic handler.
So these features need to be replaced by their own crates, if available.
The crates that provide those features in this case are called the '\lstinline{pulpissimo_hal}' and '\lstinline{pulpissimo_pac}'.
They act as an abstraction layer over and provide an interface to interact with the hardware.
\\
A minimal example of a running program can be found in \ref{code:min_example}.
This minimal example doesn't use any features from the '\lstinline{pulpissimo_pac}' directly, which is why it is not 'included'.
Also, the endless loop at the end is replaced by the '\lstinline{exit}' function from the '\lstinline{pulpissimo_hal}' crate.
This only matters if the program is running on simulated hardware. When running the program on real hardware, the '\lstinline{exit}' function is
for all intents and purposes equal to an endless loop \cite[Ch 4.3.10]{rust_pulp}.
This example also uses the '\lstinline{println}' macro from the '\lstinline{pulpissimo_hal}' crate to write messages
to standard-out (stdout), when running in a simulation, or send them using a universal asynchronous receiver-transmitter (UART)
to be viewed in software like minicom, if the program is running on real hardware.

\begin{lstlisting}[style=colorEX,language=Rust,caption={Minimal example of a program running on the pulpissimo hardware},label={code:min_example}]
#![no_main]
#![no_std]

use riscv_rt::entry;
use panic_halt as _;
use pulpissimo_hal::{exit, println};

#[entry]
fn main() -> ! {
    println!("Hello, world!");

    exit(0)
}
\end{lstlisting}


\section{Microphone technology}

A microphone consists of a diaphragm and a transducer.
Sound waves cause the diaphragm to vibrate, and
the transducer turns those vibrations into an electrical signal.
This section will introduce one common way to send audio data to different
components of a system as well as methods to encode that analog electrical
signal into a more easily processable, digital representation.

\subsection{I2S}

The inter-IC sound ($I^2S$) \cite{i2s} bus is a serial link that has been developed especially for digital audio.
It uses three lines, one serial data line (SD), which is used to transmit data, one continuous serial clock line (SCK),
which is used as the clock, and one word select line (WS).
$I^2S$ can be used in different configurations, but the master always provides WS and SCK.

\subsubsection{Serial Data}

The data consists of signed integers (in two's complement).
Due to possible differences between the word lengths and neither the transmitter nor the
receiver having any knowledge about the capabilities of the other, the 'most significant bit' (MSB) is transmitted first.
This is the bit with the highest value in a binary number, usually it is the bit in the left-most position,
while the 'least significant bit' (LSB) is located on the right end of the binary sequence.
In the transmission, the MSB has a fixed position, which means that the position of the LSB
is variable due to possible size differences between the word lengths of transmitter and receiver,
and it being dependent on the word length.\\
So when the word length of the system exceeds that of the transmitter, the LSBs are truncated
for the transmission as demonstrated in \ref{fig:truncation}.
If the receiver receives more bits than its word length, everything after the LSB
of the receiver is ignored, and if the receiver gets fewer bits than its word length,
the missing bits are set to 0.\\
Furthermore, data sent by the transmitter can either be synchronized with a trailing or leading edge
of the clock signal, however received data must be latched onto the leading edge.


\begin{figure}[htb]
    \centering
    \includegraphics[width=0.9\textwidth]{figures/fundamentals_truncation.png}
    \caption[Illustration: Truncation of words]{Example of a longer word being split into two with the lower bits being set to zero}
    \label{fig:truncation}
\end{figure}

\subsubsection{Word Select}

Word select determines the transmission channel, where 0 means the left channel and 1 means the right one.
It can be changed on a trailing or leading edge of the clock signal and the WS line changes one clock period
before the MSB is transmitted.
For the purpose of what I do in this thesis, WS is not relevant and will be tied to ground unless mentioned otherwise.

\subsection{Pulse Density Modulation}

Pulse Density Modulation (PDM) is one way to represent an analog signal with a binary signal.
The density of high/low signals at a given sampling rate encodes the state of the analog signal.
So the analog signal is encoded using only 1 bit at a high sampling rate.\\
Therefore, a constant bit stream of 1s would represent that the amplitude of the analog signal is maxed out,
while a constant bit stream of 0s would represent that the amplitude of the analog signal is at its lowest value.
Alternating 1s and 0s represent an amplitude of exactly 0.\\\\
While many digital audio systems use Pule Code Modulation (PCM), which, in contrast to PDM, uses multiple bits to represent a signal,
PDM is used a lot in mobile phones \cite{pdm_utexas} due to its simplicity and low cost.
PDM is similar to '1-bit-PCM', however a one bit wide PCM encoded signal would be much too noisy to be useful.
So to counteract the noise increase of only using 1 bit the signal is 'over sampled', thereby increasing the bandwidth of the system.
The new spectrum that is created by oversampling the signal has such a high frequency that it is out of audible range of the human ear.
'Noise Shaping' can then be used to 'push' noise into that new spectrum, thus removing it from the audible signal \cite{pdm_texas}.

\cleardoublepage

\ifthenelse{\boolean{english}}
	{\chapter{Related Work}}
	{\chapter{Stand der Technik}}
\label{cha:related_work}
% =============================================================================
% FILE NAME : 02_related_work.tex
% DEPARTMENT: University of Tuebingen
% AUTOR     : Tom Schammo
% =============================================================================
% CONTENT   : Include for chapter "Related Work"
% =============================================================================

\cleardoublepage

\ifthenelse{\boolean{english}}
	{\chapter{Concept}}
	{\chapter{Konzept}}
\label{cha:concept}
% =============================================================================
% FILE NAME : 03_concept.tex
% DEPARTMENT: University of Tuebingen
% AUTOR     : Paul Palomero Bernardo & Konstantin Lübeck
% =============================================================================
% CONTENT   : Include for chapter "Concept"
% =============================================================================
Mit das Wichtigste natürlich!

Hier wird der eigene Ansatz vorgestellt. Der Titel sollte natürlich nicht einfach Konzept heißen, sondern konkret den eigenen Ansatz benennen.


\cleardoublepage

\ifthenelse{\boolean{english}}
	{\chapter{Results and Discussion}}
	{\chapter{Ereignisses und Diskussion}}
\label{cha:results_and_discussion}
% =============================================================================
% FILE NAME : results_and_discussion.tex
% DEPARTMENT: University of Tuebingen
% AUTOR     : Tom Schammo
% =============================================================================
% CONTENT   : Include for chapter "Results and Discussion"
% =============================================================================


\subsection{Microphones}

\section{Implementation}

\section{Experiments}


\subsection{UltraTrail}

To implement the driver for UltraTrail, it first has to be added to the SVD file in the PAC
as a peripheral to generate the necessary Rust code to access hardware registers.
Subsequently, the driver can be implemented in the HAL crate.

\section{Implementation}

\section{Experiments}




% Hier sollen die erreichten Ergebnisse vorgestellt werden.
% Hierzu zählt die Vorstellung des Versuchsaufbaus sowie die geeignete Aufbereitung und Diskussion der Ergebnisse.
% Mehrwert oder Nutzen benennen!

\cleardoublepage

\ifthenelse{\boolean{english}}
	{\chapter{Conclusion and Future Work}}
	{\chapter{Zusammenfassung und Ausblick}}
\label{cha:conclusion_and_future_work}
% =============================================================================
% FILE NAME : 05_conclusion_and_future_work.tex
% DEPARTMENT: University of Tuebingen
% AUTOR     : Tom Schammo
% =============================================================================
% CONTENT   : Include for chapter "Conclusion and Future Work"
% =============================================================================

\section{Summary}

The goal of this thesis was to implement keyword-spotting on the PULPissimo and to add a Rust implementation of the UltraTrail drivers.
First, the UltraTrail AI accelerator was added to the PAC and the drivers were implemented in the HAL crate.
A test program for the driver implementation has also been ported to Rust for testing purposes.
The PDM and $I^2S$ drivers were then tested, which revealed that the chosen $I^2S$
microphone is not compatible with the PULPissimo SoC and that the PDM drivers are not fully functional.
Time has been invested in completing the PDM driver, and some bugs have been fixed, but the driver still does not work properly.
The UltraTrail driver on the other hand works as expected and passed the test program.

\section{Future Work}

Since the $I^2S$ microphone turned out to be incompatible, there have been no tests to see if the $I^2S$
driver suffers from the same problems as the PDM driver, so investigating this may also warrant future work.
There are also newer iterations of the UltraTrail AI accelerator to which the driver could be ported
that then could be tested as well.
In addition, it might be interesting to find out how to use the PDM microphone's power-down mode and
do performance, accuracy and power-consumption comparisons between different microphones.
Finally, since the driver does not work, keyword-spotting could not be implemented,
but if the PDM driver were fixed, the entire UltraTrail pipeline would be running on the development board for the first time.

\cleardoublepage

\end{onehalfspace}
% =============================================================================
% Bibliography
% =============================================================================
\ifthenelse{\boolean{english}}
	{\printbibliography[title={References},heading=bibintoc]}
	{\printbibliography[title={Referenzen},heading=bibintoc]}
\cleardoublepage
% =============================================================================
% List of abbreviations
% =============================================================================
\ifthenelse{\boolean{english}}
{
	\addcontentsline{toc}{chapter}{List of Abbreviations}
	\chapter*{List of Abbreviations}
}
{
	\addcontentsline{toc}{chapter}{Abkürzungsverzeichnis}
	\chapter*{Abkürzungsverzeichnis}
}
\begin{tabbing}
\textbf{FACTOTUM}\hspace{1cm}\=Schrott\kill
\textbf{CPU}\>Central Processing Unit
\\
\textbf{GPU}\>Graphics Processing Unit
\\
\textbf{RAM}\>Random Access Memory
\\
\textbf{MMU}\>Memory Management Unit
\\
\textbf{CISC}\>Complex Instruction Set Computer
\\
\textbf{RISC}\>Reduced Instruction Set Computer
\\
\textbf{SoC}\>System-on-a-Chip
\\
\textbf{UART}\>Universal Asynchronous Receiver-Transmitter
\\
\textbf{PDM}\>Pulse Density Modulation
\\
\textbf{PCM}\>Pulse Code Modulation
\\
\textbf{I2S}\>Inter-IC Sound
\\
\textbf{$V_{DD}$}\>Supply Voltage
\\
\textbf{GND}\>Ground
\\
\textbf{DOUT}\>Data Out
\\
\textbf{SEL}\>(Channel) Select
\\
\textbf{BCLK}\>Bit Clock
\\
\textbf{LRCLK}\>Right/Left clock, also known as Word Select
\\
\textbf{WS}\>Word Select
\\
\textbf{DAT}\>PDM Data OUT
\\
\textbf{CLK}\>PDM Clock
\\
\textbf{ISA}\>Instruction Set Architecture
\\
\textbf{ML}\>Machine Learning
\\
\textbf{AI}\>Artificial Intelligence
\\
\textbf{TCN}\>Temporal Convolutional Network
\\
\textbf{CNN}\>Convolutional Neural Network
\\
\textbf{LSTM}\>Long Short-Term Memory
\\
\textbf{RNN}\>Recurrent Neural Network
\\
\textbf{GRU}\>Grated Recurrent Unit
\\
\textbf{OPU}\>Output Processing Unit
\\
\textbf{WMEM}\>Weight Memory
\\
\textbf{BMEM}\>Bias Memory
\\
\textbf{FMEM}\>Feature Memory
\\
\textbf{LMEM}\>Local Memory
\\
\textbf{IoT}\>Internet of Things
\\
\textbf{MSB}\>Most Significant Bit
\\
\textbf{LSB}\>Least Significant Bit
\\
\textbf{OS}\>Operating System
\\
\textbf{MCU}\>Micro-Controller
\\
\textbf{PAC}\>Peripheral Access Crate
\\
\textbf{HAL}\>Hardware Abstraction Layer
\\
\textbf{SVD}\>System View Description
\\
\textbf{VA}\>Voice Assistant
\\
\textbf{JIT}\>Just-In-Time compiler
\\
\textbf{JVM}\>Java Virtual Machine
\end{tabbing}
\cleardoublepage

% =============================================================================
% List of figures
% =============================================================================
\small\normalsize
\addcontentsline{toc}{chapter}{\listfigurename}
\listoffigures
\small\normalsize
\cleardoublepage

% =============================================================================
% List of tables
% =============================================================================
\small\normalsize
\addcontentsline{toc}{chapter}{\listtablename}
\listoftables
\small\normalsize
\cleardoublepage

% =============================================================================
% "I did this myself"
% =============================================================================
\thispagestyle{empty}
\section*{Selbständigkeitserklärung}
Hiermit erkläre ich, dass ich diese schriftliche Abschlussarbeit selbständig verfasst habe, keine anderen als die angegebenen Hilfsmittel und Quellen benutzt habe und alle wörtlich oder sinngemä{\ss} aus anderen Werken übernommenen Aussagen als solche gekennzeichnet habe.
\vskip 3cm
Ort, Datum	\hfill Unterschrift \hfill
% =============================================================================
\end{document}
% =============================================================================
% DOCUMENT END
% =============================================================================
